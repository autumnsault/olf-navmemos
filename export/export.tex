%\documentclass[12pt]{olfmemo}
\documentclass[preprint,twocolumn,5p]{elsarticle}

\date{\today}
%\docnumber{2021-01}
%\date{November~19,~2019}

\usepackage{natbib}
\usepackage{hyperref}

\def\UrlBreaks{\do\/\do-} % break citation URLs at hyphens so we don't get overfull hboxes
\bibliographystyle{plainnat}

\defcitealias{Bernstein1997}{\textit{Bernstein}}
\defcitealias{DefenseDistributed2016}{\textit{Defense Distributed}}
\defcitealias{NYTvUS}{\textit{New York Times vs.\ the United States}}
\defcitealias{StaggPC2019}{\textit{Stagg \textsc{p.c.} v.\ the Department of State}}
\defcitealias{Karn1999}{\textit{Karn}}

\begin{document}

\title{An engineer's history of \textsc{us} and multilateral export controls and their application to the modern space industry}

\author{Juno~Woods\corref{cor1}\fnref{fn1}}
\ead{juno@translunar.io}
\address{Open Lunar Foundation, 399 Webster Street, San Francisco, \textsc{ca} 94117, United States}

\cortext[cor1]{Corresponding author}
\fntext[fn1]{Director of Engineering Research \& Strategy, Open Lunar Foundation}

\begin{abstract}
Export controls represent a critical barrier in the space industry, one which is poorly understood by most engineers. These laws and regulations, written long before there existed a commercial space industry, have impacted both the industry and international collaboration enormously. Moreover, they contradict a fundamental tenet for many engineers --- that their work is a form of creative expression and thus, in the United States, worthy of First Amendment protections. In this manuscript, I present a history of multilateral arms control regimes, focusing on non-binding agreements such as \textsc{cocom}, the Wassenaar Arrangement, and the Missile Technology Control Regime, their relationships with \textsc{itar} and \textsc{ear} in the United States, and the historical context under which these policies were developed. Next, I discuss export control violations in the 1990s involving Hughes and Space Systems/Loral which exerted an outsized influence on the aerospace's attitude toward \textsc{itar} and \textsc{ear}, the economic damage dealt to the commercial space industry, and the subsequent series of reforms that led to the modern-day regime. Third, I explore the jurisprudence around these regulations and laws, particularly as they relate to the First Amendment and the activities of engineers. Finally, in the context of the New Space revolution, I recommend changes to these policies that encourage a sustained and peaceful presence in space, with an eye toward cultures of innovation (e.g.\ open source), international standardization, and domestic competitive advantage.
\end{abstract}

\begin{keyword}
export controls \sep \textsc{itar} \sep \textsc{ear} \sep \textsc{cocom} \sep Wassenaar Arrangement \sep Missile Technology Control Regime \sep First Amendment \sep China \sep open source software \sep open hardware \sep open standards
\end{keyword}

\maketitle

\section{Introduction}
The set of laws, regulations, and international agreements known as export controls touch nearly every aspect of the civil space industry: hiring, academic publishing, international collaboration, even workplace architecture. While the costs of such policies were minuscule next to government-era space projects like Apollo and the Space Shuttle, the regulatory burden is more significant for modern `New Space' companies such as Space\textsc{x}, Rocket Lab, and Astra. Recent international agreements such as the Artemis Accords push for international cooperation in humanity's quest for lunar settlement, yet fail to address international arms control regimes that limit such cooperation.

In this article, I attempt to vertically integrate three aspects of export controls. I begin by exploring the history of the various formal and informal agreements that generally aim to balance conflict prevention with peaceful use. Next, I discuss the evolution of \textsc{us} laws and regulations which often exceed the requirements of such multilateral arms control regimes; I consider the Cold War terroir first, and secondly, the more recent historical developments, along with practical implementation issues amid evolving sociological and information technologies. Finally, I discuss some of the \textsc{us} jurisprudence around export controls, primarily in their intersection with free speech, as I encounter many engineers who view their work as expressive in the First Amendment sense.

It is my hope that this work will offer aid in several key areas. First and foremost, to consider changing a pre-existing system, one must understand its origins and underlying motivations; and international collaboration in the space domain requires some policy shifts. Secondly, I hope to provide New Space companies with insights into export control that aren't written by attorneys (whose roles often revolve around reducing legal risks, not identifying opportunities). Thirdly, I aim to equip proponents of collective invention in the space industry with tools to advocate for open source, open hardware, and other types of pre-competitive collaboration.

\section{The Cold War and arms control regimes}\label{sec:coldwar}
While wartime export controls were common throughout early United States history, the 1940 Export Control Act represented the nation's first peacetime trade controls (albeit looking down the barrel of war). The law included items of strategic military importance as well as commercial items, mirroring the modern regulatory distinctions between `defense articles' and `dual-use' items. The United States joined the Second World War shortly thereafter, and the Act was extended and updated periodically.\footnote{Extensions in 1944, 1945, 1946, and 1947; re-enactment in 1949 with further extensions in 1951, 1953, 1956, and 1958.} It provided broad authority to the executive branch to set penalties, issue export licenses, and determine the contents of the control lists. Moreover, it exempted the rule-making procedures from most common forms of public and judicial review. It also governed technical data. \citep{NAP1987}

Following World War~\textsc{ii}, the United States pushed Western Europe for a multilateral agreement on export controls over the period between 1945 and 1949 \citep{Yasuhara1991}. Parties to this informal agreement were collectively known as the Coordinating Committee on Export Controls, or \textsc{cocom}.

Like the Export Control Act, \textsc{cocom} had dual objectives. First, it aimed to strategically prevent equipment for manufacturing armaments from flowing to communist nations, and secondly, it attempted to impose an ``economic `iron curtain''' as described in \textsc{nsc}~68 \citep{NSC68}. This arrangement included at least the \textsc{us}, the \textsc{uk}, France, Belgium, the Netherlands, Denmark, Canada, Luxembourg, and Germany \citep{Yasuhara1991}, though accounts vary as to the exact membership at the time of founding, and others joined over time.

\textsc{cocom} generally required unanimity to add an item to its control list, and countries wishing to export controlled items agreed to seek permission from fellow \textsc{cocom} members. The three lists the organization maintained were known as the Atomic List, the Munitions List, and the Industrial List \citep{Evans2014}. While \textsc{cocom} never published these lists, nations often copied them nearly verbatim in setting their own export controls.\footnote{The British versions from 1954 onward are available at \url{https://evansresearch.org/cocom-lists/} \citep{Evans2015}.} The \textsc{us} first regulated exports to Soviet Bloc countries in late 1948 \citep{Aoi2016}, but the direction of information flow (whether to \textsc{cocom} from the \textsc{us} or the other way around) is unclear, and regulatory authority was first granted to the Commerce Department for exports in early 1949 by the Export Control Act.\footnote{The \textsc{us} lists were said to be broader than the \textsc{cocom} lists \citep{NAP1987}.} % https://www.loc.gov/law/help/statutes-at-large/81st-congress/session-1/c81s1ch11.pdf

By the sixties, anxieties about nuclear proliferation had grown in the minds of the public. While nuclear bombs were brought to bear in World War~\textsc{ii}, the invention of space launch technology in 1957 by the Soviet Union (Sputnik) and in 1958 by the United States (Explorer~\textsc{i}) enabled these devices to be delivered ballistically, magnifying fears. In the third Nixon--Kennedy presidential debate in 1960, Senator John~F.~Kennedy expressed concern ``that 10, 15, or 20 nations will have a nuclear capacity, including Red China, by the end of the Presidential office in 1964'' \citep{NixonKennedy3rd1960}.

As such, the years 1965--1968 saw the negotiation of the Treaty on the Non-Proliferation of Nuclear Weapons or \textsc{npt}. The central bargain of the \textsc{npt} was that non-nuclear-weapon states agree not to acquire nuclear weapons in exchange for the use of peaceful nuclear technology provided by nuclear-armed states. A key theme emerging from this period was the inherent challenge with all dual-use technologies, that their use or misuse depends often on the intent of those possessing them. The \textsc{npt} was the first of three binding treaties on weapons technology, the others being the Biological Weapons Convention of 1972 and the Chemical Weapons Convention of 1993 \citep{Beck2019} (outside the scope of this article). Yet the weaknesses in binding treaties would soon become apparent, owing in part to the rapid evolution of technology; less formal soft law arrangements like \textsc{cocom} were far simpler to update over time.

%By 1969, the commercial sector had begun to surpass the military in several key technological areas. Under detente, the 1969 revision of the Export Control Act was known as the Export Administration Act (\textsc{eaa}). While the National Security Council recommended bringing the \textsc{us} control list mostly in line with the \textsc{cocom} list, the Nixon Administration partially blocked the effort. \citep{NAP1987}

The creation of another such non-binding multilateral export control regime, the Nuclear Suppliers' Group (\textsc{nsg}), was sparked by the 1974 testing of a `peaceful nuclear device' by India. India had obtained a \textsc{candu} nuclear reactor and heavy water from Canada and the United States through President Eisenhower's 1953 `Atoms for Peace' program, an attempt to emphasize the peaceful uses of nuclear technology amid concerns about the nuclear arms race \citep{Walker2001}. India, never having signed the \textsc{npt}, was not bound by the treaty, and a need was seen for a supplier-side agreement to require acceptance of International Atomic Energy Agency safeguards before exporting to any non-nuclear-weapons state \citep{Burr2014}. So-called `full scope' safeguards (on the entire fuel cycle) would not be realized by the \textsc{nsg} until after the Gulf War in the 1990s \citep{Anthony2007}, demonstrating the adaptability of such non-binding arrangements.

The International Traffic in Arms Regulations (\textsc{itar}) originated in the 1976 Arms Export Control Act (\textsc{aeca}; not to be confused with the earlier \textsc{eca}s), which gave the executive branch the authority to regulate ``defense articles (arms, ammunition, and implements of war), defense services, and directly related technical data.'' While the 1968 Foreign Military Sales Act authorized foreign aid in the form of defense services, the \textsc{aeca} was the first to \textit{limit} the provision of defense services, and Congress left definition of this term up to the executive branch.\footnote{Today, \textit{defense services} are defined in 22~\textsc{cfr}~120.9(a) as
\begin{quote}
(1) The furnishing of assistance (including training) to foreign persons, whether in the United States or abroad in the design, development, engineering, manufacture, production, assembly, testing, repair, maintenance, modification, operation, demilitarization, destruction, processing or use of defense articles;

(2) The furnishing to foreign persons of any technical data controlled under this subchapter...whether in the United States or abroad;
\end{quote}
and a third clause on military training.}
The Department of State was responsible for administering these regulations, though the Defense Department generally determined the contents of the control list \citep{NAP1987}, which is known as the \textsc{us} Munitions List, or \textsc{usml}. From passage of the 1976 \textsc{aeca} onward, the \textsc{usml} included all items listed in the \textsc{mtcr} annex except those separately regulated as dual-use items.

In 1969, amid the cooling of tensions known as detente, some in the \textsc{us} had hoped to deregulate trade between the East and West, and bring the \textsc{us} control lists in line with \textsc{cocom} in the 1969 Export Administration Act (\textsc{eaa}) \citep{NAP1987}.\footnote{The \textsc{eaa} was itself a revision of the earlier \textsc{eca}.} The Export Administration Regulations (\textsc{ear}) were ultimately created by the 1979 \textsc{eaa} revision, and have generally been described as more complex than \textsc{itar}. The \textsc{eaa} authorized Commerce Department controls with several different justifications, all of which fell under the umbrella of dual-use technologies. Firstly, national security items were largely drawn from \textsc{cocom}. Secondly, it permitted controls advancing foreign policy goals. Thirdly, controls might be used to ensure \textsc{us} access to resources. Interestingly, the regulations included a general license for published, scientific, or educational technical data, meaning that no export license was required for these data \citep{NAP1987}. This exemption for information already available to the public remains in place today.

The eighties saw the creation of two new supply-side multilateral export control regimes, supplementing \textsc{cocom} and the Nuclear Suppliers Group. The Australia Group (1985) dealt with chemical weapons, and is not discussed further in this article. The fourth, the Missile Technology Control Regime, was formed in 1987, and is especially impactful upon space technology. Whereas the \textsc{nsg} focused on nuclear technology, the \textsc{mtcr} attended to the delivery technology.

The Missile Technology Control Regime was a Reagan administration response to apprehensions relating to several missile and rocket tests, by South Korea, Iraq, and India, among others \citep{Scheffran1992}. The purpose was to ``reduce the risks of nuclear proliferation by placing controls on equipment and technology transfers which contribute to the development of unmanned, nuclear-weapon delivery services'' \citep{Fialka1987}, though later it expanded to include weapons of mass destruction generally. The \textsc{mtcr} had seven founding members, and has grown to thirty-five today. All of the items regulated by the \textsc{mtcr} have been incorporated into the \textsc{usml}. \footnote{According to \textsc{itar}, 22 \textsc{cfr} Sec.~120.29.}

The end of the Cold War required a retargeting of the supply-side export control agreements away from the former Warsaw Pact states. 1993 saw the termination of \textsc{cocom}, and its replacement by the Wassenaar Arrangement (\textsc{wa}) in 1996. The \textsc{wa} differed from \textsc{cocom} in several key respects. Firstly, the membership was significantly larger than that of \textsc{cocom}, making consensus difficult. Secondly, it offered greater transparency as compared to \textsc{cocom} (including a website). Thirdly, it lacked the power to veto exports of controlled items, for which member states previously had to seek out authorization (though neither were \textsc{cocom}'s lists binding). The \textsc{wa} has been described as the weakest and least effective of the major multilateral export control regimes. Today, it controls many dual-use items, such as commercial communications and imaging satellites.

In order to understand \textsc{itar} and \textsc{ear}, it's important to understand the motivations behind the arms control agreements that undergird these regulatory frameworks. \citet{Lipson1999} argued that the arms control regimes relate to shared identity and shared norms between member states. \citet{Joyner2004} called the regimes `security communities,' though \citet{Beck2019} point out that regime members --- unlike those of other security communities --- don't necessarily view the threat of force against one another as unthinkable. \citet{Abbott2000} distinguished between `hard' and `soft' law, arguing that these regimes are soft law, which are less threatening to nations' senses of national security and sovereignty, and more adaptable than treaties.

\section{The evolution of modern \textsc{us} export controls}
A key theme following the Cold War was that of growing technical capabilities of academia and private industry, whereas previously many of the controlled exports were much more closely tied to the \textsc{us} government. Space technologies, particularly communications satellites, bounced back and forth between the \textsc{usml} and the \textsc{ccl} (Commerce Control List) several times over the subsequent two decades. With rocket technology still largely the domain of governments, inexpensive launches were in high demand among \textsc{us} companies, and this need often caused satellite manufacturers to turn to China.

Yet with the end of the Cold War, \textsc{us} national security concerns began to shift from the former Soviet Union to China. \citet{Zinger2015} has provided an excellent history of the policy aspects of export controls from the 1990s to around 2015, which begins with the explosion of three Chinese rockets carrying \textsc{us} commercial communications satellites. Companies such as Space Systems/Loral and Hughes were eager for a low-cost path to orbit, and believed they had found this in China Great Wall Industry Corporation (\textsc{cgwic}), a subsidiary of the \textsc{prc}'s main space contractor, China Aerospace Science and Technology Corporation (\textsc{casc}).

\subsection{Hughes Optus \textsc{b2} and Apstar 2}
The story begins when Chinese Long March~\textsc{2e} rockets carrying \textsc{us} satellites --- the Hughes Optus~\textsc{b2} on 21 December 1992 and the Hughes Apstar~2 on 26~January 1995 --- experienced launch failures. Investigations by Hughes pointed to the fairings as the source of the accidents. According to the 1999 Cox report, confusion over jurisdiction and licensing requirements on the part of both Hughes and government officials led to Hughes relaying technical assistance for improving Long March fairings to China after both failures. In 1995, for example, the fairing --- being a piece of the rocket --- was regulated under \textsc{itar}, but a Commerce official, assuming it to be part of the satellite, mistakenly approved the disclosure. \citep{Cox1999} % Chapter 5

The nature of the technical data Hughes provided to the People's Republic of China after the first failure makes for interesting reading. Hughes could not obtain insurance for launches subsequent to Optus~\textsc{b2} without a technical solution. The \textsc{prc} was unwilling to acknowledge that the fairing was the cause, allegedly for political reasons. The approved but unlicensed transfer included two simple recommendations for changes to the fairing: ``Add a bracket or block to prevent any possibility of overlap of the two fairing halves,'' and ``Increase the strength of the rivets along the separation line'' (which would prevent the fairing from opening prematurely). What Hughes officials viewed as fixes to design flaws, the government contended were improvements. In the process, Hughes likely revealed analysis methodologies. \citep{Cox1999} % Chapter 5

\subsection{Loral Intelsat 708}
On February 15, 1996, a similar fate befell the Loral Intelsat~708, aboard a Long March~\textsc{3b}. \textsc{cgwic} asked a Loral official to chair an independent review committee. The official recruited experts from several \textsc{us}, German, and British aerospace companies, including Hughes. The \textsc{prc} reported a broken wire in the inertial measurement unit (\textsc{imu}) as the cause. The committee disagreed with the finding, however, and sent --- without seeking \textsc{us} government review --- a draft report suggesting two other possible causes, the second of which the \textsc{prc} found to be the failure source. In essence, the disclosure led to the \textsc{prc} discovering and correcting a failure in the Long March~\textsc{3b} guidance platform. \citep{Cox1999}
% (the \textsc{imu} follow-up frame, or ``an open loop in the feedback path''). This report led to \textsc{prc} engineers ruling out the original explanation and identifying the follow-up frame as the cause, and in essence, to \textsc{prc} discovering a failure in the Long March~\textsc{3b} guidance platform. \citep{Cox1999}

Whereas the details of the Hughes launch failure disclosures were nuanced, the Intelsat review committee's alleged violation was straightforward and egregious. The Cox Report notes that ``Loral was aware from the start of the Independent Review Committee's meetings that it did not have a license for the Independent Review Committee activity'' (p. 109). The 200-page report included short-term and long-term recommendations. A \textsc{us} government interagency team noted particular concern around the exposure of Western diagnostic procedures to China. \citep{Cox1999}
%Moreover, the committee caucused in a Beijing hotel for three hours --- a conversation that was likely to have been secretly recorded by the \textsc{prc}. \citep{Cox1999}

Much of the information shared with China was in the public domain at the time, leading Loral officials to believe that no license was required for these technical data. However, the Department of Defense believed that the review committee performed a ``defense service.''
\begin{quote}
In general, a \textsc{us} citizen may transfer public domain information to a foreign national. However, such a transfer is not allowed if it occurs in the performance of a defense service, which is defined in Part 120 of the International Traffic in Arms Regulations. (p.~164)
\end{quote}
Moreover,
\begin{quote}
The expertise and experience of the person making the disclosure, and the circumstances of the disclosure, are important in determining whether a defense service has been performed through such a disclosure. As an example, simply giving a foreign national an article from the Encyclopedia Britannica is not an export requiring a license. If, however, the article is provided to a foreign national by an experienced engineer in the context of specific technical discussions, a defense service that requires a license may have been performed. \cite{Cox1999}
\end{quote}

The defense service claim is particularly interesting given that China's ballistic missiles were already sufficiently advanced that neither the \textsc{cia} nor \textsc{dod} believed the committee's improvements would benefit the program; China used a different guidance system and \textsc{imu} aboard its missiles \citep{Cox1999}. While the \textsc{us} had legislated (and regulated) above and beyond the requirements of the \textsc{mtcr} previously, in this case they had taken enforcement just as seriously. While the \textsc{us} was also concerned about analysis methodologies and engineering approaches being shared with the Chinese, these also were not covered in the \textsc{mtcr}.

The government charged Loral with 64 counts \citep{Marquis2002}. In a settlement, the company paid fines of \$20\textsc{m}. Hughes' fines dwarfed these at \$32\textsc{m}, perhaps because the failures had been systemic rather than at the level of an individual employee.

\subsection{Fallout and subsequent years}
A month after the Long March~\textsc{3b} crash (and well before the formation of the independent review committee), on 14 March, the Clinton administration announced that licensing authority for commercial communication satellite exports would be shifted from from State to Commerce, in this case to incentivize China to participate more fully in the \textsc{mtcr}.\footnote{The text of the order is available in \citet{State61FR56894_1996}.} As \citet{Zinger2015} noted, the shift was short-lived; following the release of the Cox Report in 1999, Congress returned commercial satellite licensing to \textsc{itar}, and took away the president's power to determine satellite jurisdiction, where it remained for fourteen years. The United States became the only nation which treated all commercial satellites as munitions \citep{Section1248}.

In 1997, the \textsc{us} had enjoyed a substantial majority of the satellite manufacturing market. Ten years later, it had lost that dominance, with foreign satellite manufacturers selling products they advertised as `\textsc{itar}-free.' The Defense Department found that \textsc{us} companies had lost \$2.35\textsc{b} of sales due to the \textsc{itar} licensing process. \citep{AFRL2007}

It took another six years for Congress to stem the flow. The Obama Administration proposed reforms to the \textsc{usml} with the manufacturing and technology sectors in mind in 2010, and a 2012 report by the State and Commerce Departments affirmed many of those reforms, acknowledging that many of the satellite technologies regulated under \textsc{itar} were available without such stringent controls from a number of other countries \citep{Section1248}. Goals of these reforms included unification of controls on a single list, under a single licensing agency, with only one agency handling enforcement \citep{Fergusson2020}.

In 2013, Congress finally returned to the executive branch the power it had taken in 1999, except as regarding China, North Korea, and state sponsors of terrorism. \citet{Zinger2015} argued that the 2013 reforms were both much-needed for the \textsc{us} commercial space industry and an abject failure in terms of their ability to prevent confusion about the split jurisdiction of certain exports, such as in the Hughes case.

The Export Control Reform Act (\textsc{ecra}) of 2018 recognized economic security as a key goal.
\begin{quote}
The national security of the United States requires that the [\textsc{us}] maintain its leadership in the science, technology, engineering, and manufacturing sectors, including foundational technology that is essential to innovation. Such leadership requires that United States persons are competitive in global markets. The impact of the implementation of this part on such leadership and competitiveness must be evaluated on an ongoing basis and applied in imposing controls...to avoid negatively affecting such leadership. \citep[Section 1752(3)]{ECRA2018}
\end{quote} 
It also noted the ineffectiveness of unilateral export controls on items readily available abroad. Thirdly, the law emphasized the importance of small and medium-sized businesses \citep{ECRA2018}. Finally, the law further expanded executive power with regards to dual-use exports \citep{Fergusson2020}. While these changes to federal law created the framework for regulatory improvements, only some of these have been realized.

While not specifically related to export controls, Congress further limited collaborations between the \textsc{us} and China in 2011, via an amendment by Rep.~Frank Wolf, over espionage \citep{Pentland2011} and human rights concerns --- a policy which remains in force \citep{Foust2019}. The law required Congress to specifically authorize any interactions that involved the \textsc{prc} among \textsc{nasa}, the Office of Science and Technology Policy, or the National Space Council. The effort does not appear to have slowed China, which has successfully landed multiple spacecraft on the Moon, including one on the far side.

Perhaps the largest impact of the modern export control regime has been on hiring. The Defense Department has noted aerospace and defense companies face a skills gap in the native-born population \citep{DoD2018}, yet most \textsc{us} aerospace job postings include a statement that applicants must be United States citizens (or at least \textsc{us} persons). In contrast, foreign-born workers made up nearly one-sixth of the labor force in 2014, and over 70\% of creative information technology roles in Silicon Valley; most were not \textsc{us} citizens \citep{Otoiu2017}.

\section{Export controls and the First Amendment}
Little has been said thus far about the relationship between export controls and the First Amendment. Among many computer scientists and cryptographers, it is practically an article of faith that engineering work is information and ``information wants to be free.''\footnote{This quotation is attributed to Stewart Brand, an influential figure in hacker circles. It has been interpreted using both definitions of `free' (cost and freedom) regardless of Brand's intentions.} Yet the courts have split on whether export controls restrict free expression in cases of national security.

While some have suggested that prohibitions of sharing of technical data like those that led to the Cox investigations are no different from government classification of sensitive information, there is a fundamental difference. Those individuals who receive security clearances generally consented to curtailing some freedoms for the sake of national security. For \textsc{us} persons dealing with export controlled materials, the question of consent is less certain. The space industry has generally steered clear of First Amendment arguments, perhaps due to its continued dependence on government funding, so I turn in this next section to a few cases outside of aerospace.

While a graduate student at the University of California, Berkeley, Daniel Bernstein developed Snuffle, an encryption algorithm. Knowing that the \textsc{usml} regulated some encryption technologies under \textsc{itar}, he asked the Department of State if he needed an export license to publish Snuffle, either in source code form or as an academic paper.
\begin{quote}
The State Department responded that Snuffle was a munition under the International Traffic in Arms Regulations...and that Bernstein would need a license to ``export'' the Paper, the Source Code, or the Instructions. There followed a protracted and unproductive series of letter communications between Bernstein and the government, wherein Bernstein unsuccessfully attempted to determine the scope and application of the export regulations to Snuffle. \citep{Bernstein1997}
\end{quote}
%
Bernstein challenged the law in court, arguing that it was a prior restraint on his First Amendment rights to free expression \citep{Bernstein1997}; the district court issued a summary judgment in his favor on those grounds. While Bernstein's challenge wound its way through the courts, the Clinton administration transferred jurisdiction of encryption from State to Commerce \citep{ExecOrder13026_1996},\footnote{This announcement was made only a day after the 1996 transfer of satellites from State to Commerce, but the executive order was delayed until mid-November.} causing Commerce to be added as a defendant. The district court again issued a summary judgment against the government and barred the Commerce Department from enforcing the relevant regulations. The government appealed, and a three-judge panel of the Ninth Circuit affirmed the district court's decisions.

The Ninth Circuit decision, as well as the district court judgments, emphasized the question of source code as expression. On the topic of prior restraint,
\begin{quote}
In \textit{Freedman v.\ Maryland}, the Supreme Court set out three factors for determining the validity of licensing schemes that impose a prior restraint on speech: (1) any restraint must be for a specified brief period of time; (2) there must be expeditious judicial review; and (3) the censor must bear the burden of going to court to suppress the speech in question and must bear the burden of proof.''%\footnote{Three cases are cited on prior restraint in \textit{Bernstein}. The first of these is \textit{New York Times v.\ United States}. The second is} 
\citep[p.~4239]{Bernstein1997}
\end{quote}
Additionally, from \citetalias{NYTvUS}, prior restraint is only justifiable on national security grounds if publication would ``surely result in direct, immediate, and irreparable damage to our Nation or its people,'' \citep{NYTvUS} as cited in \citetalias{Bernstein1997}.

Yet the court clearly indicated that not all software is expression. The decision suggested that source code is more likely to be expressive, and particularly source code expressing scientific ideas.
\begin{quote}
First, we note that insofar as the \textsc{ear} regulations on encryption software were intended to slow the spread of secure encryption methods to foreign nations, the government is intentionally retarding the progress of the flourishing science of cryptography. To the extent the government's efforts are aimed at interdicting the flow of scientific ideas (whether expressed in source code or otherwise), as distinguished from encryption products, these efforts would appear to strike deep into the heartland of the First Amendment. \citep[p.~4242]{Bernstein1997}
\end{quote}

The government appealed the decision to the full Ninth Circuit, and the \citetalias{Bernstein1997} decision was withdrawn in preparation for the rehearing. At this point, the Commerce Department rewrote the regulations \citep{EncryptionRule2000}, causing the court to declare the case moot. Bernstein's was not the only such encryption export case tabled by the new regulations. Phil Karn carefully documented his own fight with the government, in which the State Department ruled that Bruce Schneier's \textit{Applied Cryptography} textbook was in the public domain and therefore exempt from \textsc{itar}, but ultimately ruled that the accompanying disks (which contained the source code appearing in the textbook) were munitions \citep{Karn1999}.

In 2012, Cody Wilson --- while a law student at the University of Texas at Austin --- and his non-profit, Defense Distributed, released computer-aided design (\textsc{cad}) models for \textsc{3d} printing firearms as well as computer numeric control (\textsc{cnc}) milling files for producing \textsc{ar}-15 lower receivers.\footnote{The lower receiver is the portion of the firearm whose sale is regulated in the United States, but the law permits at-home manufacture.}

The State Department requested removal of these files on the grounds that they were \textsc{itar}-controlled technical data, and that posting them on the Internet constituted an export (and thus required a license). Defense Distributed sued the State Department on prior restraint grounds, requesting a preliminary injunction. In 2016, a three-judge panel of the Fifth Circuit denied the injunction, declining to address the First Amendment question. The government had a substantial interest in protecting national security; moreover, they wrote that the temporary harm to the plaintiffs of a First Amendment violation needed to be balanced against the potential permanent harm to public safety, given the irreversible nature of Internet publishing.\footnote{The \textit{Harvard Law Review} reviewed the First Amendment aspects of the \citetalias{DefenseDistributed2016} case in 2017 \citep{Harvard2017}, and offered a persuasive analysis. They agreed with the Fifth Circuit's decision but argued that the court should have rejected the injunction on grounds that \textsc{cad} files were not protected speech:
\begin{quote}
If \textsc{cad} files were to fall within the coverage of the First Amendment, the government's ability to regulate the content, safety, and use of these files would be sharply limited. Because these files define the specifications of tangible objects, the government would thus also be limited in its ability to regulate the physical world --- from houses to bioweapons.
\end{quote}} \citep{DefenseDistributed2016}

In 2016, the Commerce Department updated \textsc{ear} to include an exemption for published materials \citep{PublishedRule2016}. The rule defined technology or software as `published' as one might expect, to include materials on the Internet and fundamental research, and indicated that these were not subject to \textsc{ear}.

\textsc{itar}, too, has contained an exemption for public domain information since 1985; however, publishing technical data has generally required an export license, with exceptions carved out for fundamental research and academic publication. Numerous challenges on First Amendment grounds have been rebuffed by courts \citep{Edler1978,Posey1989,Mak2012,StaggPC2019}, which have consistently held that the government has more flexibility in regulating `content-neutral' speech than that espousing a particular viewpoint. In \citetalias{StaggPC2019}, a former contractor to the Defense Directorate of Trade Controls (which is responsible for enforcing \textsc{itar}), Christopher Stagg, sued over his law firm's right to publish educational materials on export control, which included \textsc{itar} technical data, on its website. Many of these technical data were already publicly available elsewhere on the Internet, which has never been explicitly included in the \textsc{itar} public domain exemption.

In its decision, the court found against Stagg \textsc{p.c.}, quoting the State Department's brief:
\begin{quote}
The \textsc{itar} does not require a license or other authorization to republish information that is available in printed books, newspapers, journals, and magazines that can be purchased in a physical bookstore or newsstand or checked out from a public library, because such information is already in the public domain and no longer considered \textsc{itar}-controlled technical data. The \textsc{itar} does not require a license or other authorization to publish fundamental research that meets the criteria set forth in Sec.~120.11(a)(8), nor does it require a license or other authorization to publish information concerning the general scientific, mathematical, or engineering principles commonly taught in schools, colleges, and universities, \textit{id.} Sec.~120.10(b)(1). The \textsc{itar} also does not require a license for purely domestic publication or dissemination of files. \textit{See id.} Sec.~120.17 (defining export). \citep{StaggPC2019}
\end{quote}
While the court noted that the Internet was not explicitly listed in Sec.~120.11, it agreed with the plaintiffs that the library exemption might apply to certain websites occupying an analogous role.

While courts seem less willing to weigh in on cases pitting so-called content-neutral speech against national security or other compelling governmental interests, concerns about privacy have played a role. Just before \citetalias{Bernstein1997} was decided, the government dropped charges against Phil Zimmerman, privacy activist and creator of the open source \textsc{pgp} encryption program for exporting the software \citep{Markoff1996}. Encryption and privacy, too, were at issue in the the \citetalias{Karn1999} and \citetalias{Bernstein1997} cases. Yet \citetalias{Karn1999} related to freedom of the press and \citetalias{Bernstein1997} to academic and press freedom. \citetalias{DefenseDistributed2016}, on the other hand, put the \textsc{us} at risk of violating its commitments under the Wassenaar Arrangement. While several cases involved source code, designed explicitly to be human-readable, the government might have treated schematics and computer instructions more like physical hardware.

These issues expose several weaknesses in \textsc{itar} and the Arms Export Control Act. Open source software and hardware, like academic research, is a type of collective invention --- where multiple entities work together to create something collaboratively and iteratively \citep{Allen1983,Schrape2019}. Moreover, that so much work is now conducted in the cloud makes it possible for software and schematics to be developed from the ground up in public-facing web applications (e.g.\ on GitHub). When does the work become a munition, and at what point does the act of publication occur which requires the export license?

Questions of open source in the space industry come up again and again in New Space organizations. In 2018, Consensys Space acquired Planetary Resources, a company which sought to survey and mine asteroids; Consensys released Planetary Resources' patents and pledged not to take legal action against those who used the company's intellectual property \citep{Consensys2018}. The company did not release its source code or schematics because of the cost of combing through the code for potential \textsc{itar} violations (personal communication).

Open Research Institute, a small \textsc{us}-based non-profit research and development organization, aims to develop open source software and hardware for use in space, including an open radio for ground stations and geosynchronous amateur radio satellites, and ran up against these same issues \citep{ORIBlog}. While their work appears to fall under \textsc{ear} \citep{ORIDDTC}, Open Research Institute has taken the precautionary step of posting a notice on their website, along with the published code and schematics:
\begin{quote}
Our intent is for all of this work to be ``Public Domain'' under \textsc{itar} 120.11 and ``Published'' under \textsc{ear} 734.7, and thus not subject to \textsc{itar} or \textsc{ear}.

In addition, it is \textsc{ori}'s policy not to provide services that might be restricted under \textsc{itar} or \textsc{ear}, and we do not allow participation in our projects that could be connected with the national defense of any nation. \citep{ORIStatement}
\end{quote}

As of the writing of this article, \textsc{us}-based Swift Navigation provided \textsc{gnss} receiver source code in its open source library Libswiftnav. The code included a function with the comment,
\begin{quote}
\textsc{note}: The following condition is required to comply with \textsc{us} export regulations. It must not be removed. Any modification to this condition is strictly not approved by Swift Navigation, Inc. \citep{Libswiftnav575}
\end{quote}
Presumably, this source code did not violate \textsc{itar} because it contained the velocity limitation (and above it a similar altitude restriction). However, if someone cloned the repository and removed the condition\footnote{One would probably also need to remove the maximum altitude restriction right of the `or' statement on line 570, not solely the condition indicated in the comment.} in GitHub's built-in editor and pushed commit, would this constitute an export, or would the code already be in the public domain? In fact, while writing this piece, I cloned the code and typed out the just-described change in the editor (strictly to edify readers), but ultimately did not hit commit out of fear of prosecution. Likewise, I typed the changes into this document, but ultimately removed them, again out of concern for my career as an aerospace engineer.\footnote{Let this instead serve as an example of a prior restraint.} I requested an advisory opinion from the Defense Directorate of Trade Controls (\textsc{ddtc}, which handles export licenses that fall under \textsc{itar}) on this topic last August, but have not yet received a response.

Releasing intellectual property to the public occasionally happens when an organization goes out of business or pivots to a different area of work --- but these releases are less likely in the space industry. For every example one hears of an engineer successfully arguing to release something over the course of a year \citep{Scoles2017}, one imagines there are some uncountable number of unsuccessful attempts as well.

\section{Recommendations and Conclusion}
In its hegemony, the United States has historically been the prime mover of the various arms control regimes. As the main supplier state, the \textsc{us} had as much power to accomplish its goals by restricting its own exports as by persuading allies to restrict theirs. At the time the export control system was designed, the regulations worked to restrict the flow of \textsc{us} government technology to competitor states; \textsc{us} companies, less globalized and possessing less technology than the federal government, had less to lose. Moreover, ``Dual-use technologies represent[ed] a relatively small and easily isolated category'' in 1949. \citep{Kuttner1991}

Today, the arms control landscape is quite different. North Korea, having obtained intercontinental ballistic missile (\textsc{icbm}) technology from China and nuclear warhead designs from Pakistan, has exported \textsc{icbm} technology since the 1980s \citep{Squassoni2006}. This year Davenport wrote that ``North Korea...is viewed as the primary source of ballistic missile proliferation in the world today'' \citep{Davenport2021}.

In 1991, \citet{Kuttner1991} wrote that ``the \textsc{us} export control system rests on three tacit presumptions that were more or less correct in 1949 but that were long ago overtaken by events.'' Firstly, ``The United States is the leader in, and therefore controls the diffusion of, most advanced technology.'' Secondly, ``Exports don’t matter much to the \textsc{us} economy, so the commercial costs of the system are trivial.'' And thirdly, ``Dual-use technologies represent a relatively small and easily isolated category. In the electronic era, virtually all advanced technologies have dual uses.''

%   https://www.bis.doc.gov/index.php/documents/regulations-docs/2263-legal-authority-for-the-export-administration-regulations-1/file
% ECRA of 2018: in its policy statement, 1752(2)(F): ``To facilitate military interoperability between the United States and its North Atlantic Treaty Organization (NATO) and other close allies.''
% 1752(3): `` The national security of the United States requires that the United States maintain its leadership in the science, technology, engineering, and manufacturing sectors, including foundational technology that is essential to innovation. Such leadership requires that United States persons are competitive in global markets. The impact of the implementation of this part on such leadership and competitiveness must be evaluated on an ongoing basis and applied in imposing controls under sections 1753 and 1754 to avoid negatively affecting such leadership.''
% 1752(6): ``Export controls applied unilaterally to items widely available from foreign sources generally are less effective in preventing end-users from acquiring those items. Application of unilateral export controls should be limited for purposes of protecting specific United States national security and foreign policy interests.''
% 1753(a)(6) strengthen industrial base --- really refers to the defense base, but should be amended to also include space competitiveness
% 1753(a)(7) ``enforce the controls through means such as...guidance in a form that facilitates compliance by United States persons and foreign persons, in particular academic institutions, scientific and research establishments, and small and medium-sized businesses.''
% 1754(a)(6) ``establish a process for an assessment to determine whether a foreign item is comparable in quality to an item controlled under this part and is available in sufficient quantities to render the United States export control of that item or the denial of a license ineffective, including a mechanism to address that disparity''
% 1754(a)(8) ``require and obtain such information from United States persons and foreign persons as is necessary to carry out this part''

There are several potential areas for improvement.

\textsc{us} export controls have directly benefitted adversary nation's economies at the expense of domestic entities for some time. In 1991 Kuttner pointed out, ``...Soviet-built machine tools have been shown at trade fairs in Chicago; these tools, if made by \textsc{us} tool builders, could not be exported to the Soviet Union'' \citep{Kuttner1991}. Controls on \textsc{us} companies' space technologies should not be stricter than those imposed on foreign competitors by the \textsc{mtcr} and other regimes. At the very least, exports to other regime members ought to be further de-regulated. \textsc{us} companies should have access to a level playing field relative to the space industries of \textsc{us} allies. Alternatively, funding and legal resources should be provided to small businesses in the United States that want to seek export licenses, including for the hiring of non-\textsc{us} persons, concomitant with a dramatic acceleration of the licensing process.

The regulations affecting technical data and defense services should be amended to eliminate regulatory burdens on collective invention that occurs largely in the public domain (e.g.\ open source, copyleft, and academic projects), particularly those in the space industry. Specifically, `defense services' should be defined with respect to the payer and the recipient, such that open source software and hardware projects are not at risk of prosecution or litigation. Moreover, the government should bring its views on engineering inline with the views of professional engineers, recognizing that products of engineering work may be expressive and therefore deserving of additional First Amendment protections. Such protections should not be limited to engineering work which serves an ancillary purpose such as privacy.

Thirdly (and independently of export controls), Congress should eliminate the Wolf Amendment and loosen restrictions on collaborations between Chinese and \textsc{us} space interests. International collaboration is enormously useful for generating positive sentiment between adversaries. While such collaborations between the \textsc{us} and China should be carefully considered, the current ban is too strict and has little impact on human rights (an original goal of the Wolf Amendment). On the other hand, it is possible for reforms to be overly broad. For example, there is significant money arrayed in support of reducing export controls on drones and other conventional weapons, including on the part of repressive regimes which might bring these devices to bear on their own citizens \citep{Summers2020}.

The 2018 \textsc{ecra} offered fertile grounds for a regulatory overhaul. The law explicitly indicated that export regulations should allow for sharing of technology with \textsc{us} allies such as those in \textsc{nato}, particularly as might be needed for `military interoperability' \citep{ECRA2018}. This same policy should be extended to peaceful space technologies in recognition of our obligations to render aid under the Outer Space Treaty and for interoperability and standardization under the Artemis Accords. In general, regulations should be brought inline with the legislative intent of \textsc{ecra}, which privileges economic security and national competitiveness, particularly for small and mid-sized businesses.

Moreover, the `regulations diverge from practice' (as described in personal communications with an attorney involved in writing the 2012 reforms) in some areas, and ought to be brought inline with the law and current enforcement practices. Numerous conversations with regulatory experts have indicated that \textsc{ddtc} does not consider open source software releases or open standards to need export licenses, though such releases of technical data are considered `deemed exports' in the regulations, and in some cases might meet the definition of defense services. It is important that laypeople reading the regulations be able to understand what is and isn't expected of them.

The Outer Space Treaty provides that space is the common heritage of all humankind, but by restricting access to technologies for space travel, we restrict access to this vast resource. We run the risk of incentivizing militarization of space. Moreover, while treaty calls for rescue of personnel in distress, such acts are inhibited by the absence of international technical standards (such as might be needed for compatible airlocks or $\textsc{co}_2$ scrubbers, for example). Broad restrictions on sharing of technical data act as an unnecessary barrier to standardization. Such barriers are most easily demonstrated in the contrast between the development of global navigation satellite systems and the Internet. Whereas there are around eight different competing national standards for global positioning due to \textsc{us} restrictions on the first such system, there is a single Internet, developed through an open request-for-comment process.

The plaque on Apollo~11, the first crewed spacecraft to land on the Moon, bears the inscription,
\begin{quote}Here men from the planet Earth first set foot upon the Moon. We came in peace for all mankind.\end{quote} If the Moon truly is for all humankind, technologies promoting access to space cannot be painted with the same brush as technologies for killing humans.

\section{Acknowledgements}
The author wishes to thank Chelsea Robinson, Jessy Kate Schingler, Chelsea McMahon, and Giuliana Rotola for helpful comments. This work was supported by donors of Open Lunar Foundation.

% Conclusion quotes
%
% * https://hbr.org/1991/01/how-national-security-hurts-national-competitiveness
%.  Robert Kuttner, 1991 - Harvard Business Review
%   
%  \textsc{us} export control system is stricter than CoCom requires.
%. "The U.S. export control system rests on three tacit presumptions that were more or less correct in 1949 but that were long ago overtaken by events.
%  1. The United States is the leader in, and therefore controls the diffusion of, most advanced technology...
%  2. Exports don’t matter much to the U.S. economy, so the commercial costs of the system are trivial.
%. 3. Dual-use technologies represent a relatively small and easily isolated category. In the electronic era, virtually all advanced technologies have dual uses."
% "In fact, Soviet-built machine tools have been shown at trade fairs in Chicago; these tools, if made by U.S. tool builders, could not be exported to the Soviet Union!"
% "Put simply, the U.S. government resists the idea that non-CoCom nations should have state-of-the-art telephone systems. This resistance boils down to one concern --- electronic eavesdropping."
% "And the U.S. government must cease imposing higher unilateral constraints on U.S. industry. This means not only higher fences around fewer products but also equivalent fences around all prospective exporters."

% * https://www.cnas.org/publications/reports/rethinking-export-controls-unintended-consequences-and-the-new-technological-landscape
%.  Martijn Rasser, 2020 - Center for New American Security
%   "U.S. export controls were designed for an era when the United States enjoyed overwhelming technological dominance."
%.  "To be sure, export controls still function reasonably well as it pertains to the movement of dual-use goods, and when applied multilaterally, such as through the Wassenaar Arrangement. In the case of a rising, technologically capable China, however, U.S. export controls would be more effective if wielded as part of a comprehensive national security and economic statecraft that safeguard U.S. technological advantages, rather than as a traditional non-proliferation tool."
%.  "The most effective export controls are ones tailored to undermining a foreign actor’s technology indigenization efforts. This requires understanding American technological advantages—and how to sustain and amplify them—and knowing where the technological chokepoints are."
%.  * Globalized supply chain, e.g.\ "A report by data analytics firm Govini found that the Department of Defense has dozens of Chinese companies in its IT supply chains."
%. "China’s military aviation industry took a “buy, build, or steal” approach, which included co-production and reverse engineering, to attain technological capabilities in areas subject to export controls including avionics, airframe design, fire control radars, and composite materials."
%. "Export controls are not an end in and of themselves. Investments in R&D of next-generation technologies need to happen concurrently to lay the foundation for continued competitiveness. More broadly, export controls need to be crafted as part of an overarching national strategy for technology and a broader economic statecraft strategy, not as scattershot policy actions."

% * https://www.latimes.com/business/story/2019-10-24/huawei-export-ban-us-companies-confusion
%   Leonard and King, 2019
% Effects of Huawei sanctions on \textsc{us} companies differs depending on how they interpret the law/regulations

% * https://www.armscontrol.org/factsheets/mtcr
% Davenport, Kelsey 2021
% "North Korea, for example, is viewed as the primary source of ballistic missile proliferation in the world today"

% * https://www.cfr.org/backgrounder/north-koreas-military-capabilities
% Albert, Eleanor 2020
% * North Korea not only has ICBMs capable of reaching the \textsc{us}, but it exports them, making it a nuclear supplier.

% * https://www.cfr.org/backgrounder/irans-ballistic-missile-program
% Bruno, Greg 2012
% * Iran has ballistic missiles but unclear if they can reach the \textsc{us}.

% * https://www.middleeasteye.net/fr/news/irans-priority-developing-missiles-export-allies-minister-382613695
%  2017
% Iran prioritizing export of missile technology to allies

% * https://www.pogo.org/investigation/2020/10/in-victory-for-lobbyists-trump-administration-loosens-drone-export-rules/
% 2020
% Risk of lobbyist influence

\bibliography{export}
\end{document}